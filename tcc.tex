\documentclass[
	12pt,				
	oneside,
	a4paper,
	english,
	brazil,
	]{abntex2}

\usepackage{tcc} 								% Customizacoes

% Pacotes fundamentais 
\usepackage{lmodern}								% Fonte Latin Modern
\usepackage[T1]{fontenc}							% Selecao de codigos de fonte.
\usepackage[utf8]{inputenc}						% Codificacao do documento (acentos)
\usepackage{lastpage}							% Usado pela Ficha catalografica
\usepackage{indentfirst}							% Indenta o primeiro paragrafo de cada secao.
\usepackage{color}								% Controle das cores
\usepackage{graphicx}							% Inclusao de graficos
\usepackage{microtype} 							% Melhorias de justificacao
\usepackage[alf,abnt-emphasize=bf]{abntex2cite}	% Citacoes padrao ABNT
\usepackage{lipsum}								% Geracao automatica de textos de exemplo
% ---
	
% Pacotes de citações
\usepackage[brazilian,hyperpageref]{backref}	 
\usepackage[alf]{abntex2cite}					% Citações padrão ABNT

% Informações de dados para CAPA e FOLHA DE ROSTO
\titulo{\Large{AGRUPAMENTO DE PACIENTES EM REABILITAÇÃO PÓS-COVID PARA INDICAÇÃO DE TRATAMENTOS DIFERENCIADOS}}
\autor{\textbf{Eduardo José Pereira Barreto}}
\local{Salvador (BA)}
\data{2021}

\instituicao{\textbf{Centro Universitário Senai Cimatec}}
\filiacao{ESPECIALIZAÇÃO EM DATA SCIENCE \& ANALYTICS}
\orientador{Prabhát}

\preambulo{Projeto  apresentado  ao  CENTRO UNIVERSITÁRIO  SENAI  CIMATEC como requisito parcial para obtenção do  título  de  Especialista  em  Data Science \& Analytics.}
% ---

% informações do PDF
\makeatletter
\hypersetup{
     	%pagebackref=true,
		pdftitle={\@title}, 
		pdfauthor={\@author},
    	pdfsubject={\imprimirpreambulo},
	    pdfcreator={LaTeX with abnTeX2},
		pdfkeywords={abnt}{latex}{abntex}{abntex2}{relatório técnico}, 
		bookmarksdepth=4
}
\makeatother
% --- 

% O tamanho do parágrafo é dado por:
\setlength{\parindent}{1.3cm}

% Controle do espaçamento entre um parágrafo e outro:
\setlength{\parskip}{0.2cm}  % tente também \onelineskip

% Compila o índice
\makeindex


% Início do documento
\begin{document}


\selectlanguage{brazil}

% Retira espaço extra obsoleto entre as frases.
\frenchspacing 


% ----------------------------------------------------------
% ELEMENTOS PRÉ-TEXTUAIS
% ----------------------------------------------------------
\pretextual

\imprimircapa
\imprimirfolhaderosto*
\imprimirdeclaracaoisencao

% inserir lista de ilustrações
\pdfbookmark[0]{\listfigurename}{lof}
\listoffigures*
\cleardoublepage
% ---

% inserir lista de tabelas
\pdfbookmark[0]{\listtablename}{lot}
\listoftables*
\cleardoublepage
% ---

% ---
% inserir o sumario
% ---
\pdfbookmark[0]{\contentsname}{toc}
\tableofcontents*
\cleardoublepage
% ---


% ----------------------------------------------------------
% ELEMENTOS TEXTUAIS
% ----------------------------------------------------------
\textual


% ------------------
% ELEMENTOS TEXTUAIS
% ------------------
\textual

\chapter{INTRODUÇÃO}

A atual pandemia causada pela CoViD-19 vem afetando a saúde das pessoas não só durante o período da infecção pelo vírus, mas também após a recuperação dos pacientes e fim dos sintomas associados. Muitos apresentam sequelas que impactam seu cotidiano causando limitações nas atividades da rotina diária. Diante desse cenário foi criado um centro de reabilitação para pacientes que foram infectados e apresentaram sequelas. Após o preenchimento de formulários por médicos e fisioterapeutas com dados dos pacientes, os mesmos serão utilizados para identificar abordagens específicas para pacientes que apresentam características semelhantes.
\chapter{MOTIVAÇÃO}

Diante do número crescente de pessoas com limitações adquiridas após
recuperaração de contaminação por Covid, é importante que os pacientes sejam identificados de acordo com seus sintomas atuais e históricos para que o corpo médico possa ter um indicativo inicial para uma tratamento mais específico.
\chapter{OBJETIVO}

\section{GERAL}
 
Utilizar o conceito de Inteligência Artificial (IA) com técnicas atuais de reconhecimento de padrões de forma a agrupar pacientes com um conjunto de sintomas, características e histórico similares, visando indicar uma abordagem terapêutica inicial específica para cada grupo e proporcionando uma triagem mais rápida.

\section{ESPECÍFICOS}

Identificar informações relevantes para utilização dos algoritmos de agrupamento, eliminar ou substituir dados faltantes, e transformar os dados de forma a padronizá-los para possibilitar a melhor aplicação das técnicas de reconhecimento de padrões. Após essa fase, utilizar e comparar os métodos atuais de agrupamento de forma a escolher qual o mais apropriado para obter o objetivo geral descrito acima.
\chapter{REFERENCIAL TEÓRICO}

\textit{Descrever os principais autores que versam sobre o tema escolhido. Buscar livros, jornais e revistas que tenham citações/análises importantes e atuais sobre o tema escolhido.}
\chapter{METODOLOGIA}

Para este estudo foi utilizada a plataforma Google Colaboratory, tendo como linguagem de programação a versão 3.7.11 do Python. As bibliotecas utilizadas foram:
scikit-learn (0.24.2) - algoritmos de clusterização kmeans, hierárquico e DBSCAN; método da silhueta para avaliação dos agrupamentos; e decomposição em componentes principais (PCA);
scikit-learn-extra (0.2.0) - algoritmos de clusterização kmedoids;
scipy (1.4.1) - exibição de dendrogramas para visualização de agrupamento hierárquico;
matplotlib (3.2.2) - exibição de gráficos;
numpy (1.19.5) - utilitário para funções numéricas;
pandas (1.1.5) - leitura e manipulação de dados;
seaborn (0.11.1) - exibição de gráficos para análise exploratória;
  () - ;
Os passos seguidos foram:
obtenção dos dados - as informações dos pacientes foram disponibilizadas pela Fundação José Silveira através de exportação pelo sistema RedCap;
visualização dos dados - exploração de como os dados estão armazenados, visualização de suas características, análise estatística;
transformação dos dados - seleção de colunas, preenchimento de valores nulos, exclusão de instâncias sem valores, conversão de tipos, e padronização de valores;
análise dos principais atributos e componentes - obtenção dos principais componentes, verificação da variância acumulada, e representação gráfica dos atributos;
aplicação dos algoritmos de clusterização - criação de modelos de agrupamento com as técnicas kmeans, kmedoids, dbscan e de hierarquia;
comparação dos algoritmos de clusterização - verificação do desempenho de cada técnica utilizando o método da silhueta.

\section{ANÁLISE EXPLORATÓRIA DE DADOS}
\textit{Apresentar informações relevantes do conjunto de dados, identificar comportamentos médios e discrepantes, investigar se há interdependência entre dados e procurar identificar tendências. É uma etapa importante para apresentar informações não apenas sobre características de interesse, mas diversas outras informações que auxiliam no entendimento dessa característica.}

\section{ENGENHARIA DE DADOS}
\textit{Apresentar informações a respeito dos critérios utilizados para a limpeza de dados. Houve alguma transformação nos dados, reorganização, ajuste ou combinação de dados de diversas fontes.}

\section{CONSTRUÇÃO DO MODELO DE INTELIGÊNCIA ARTIFICIAL}
\textit{Como foi construído o modelo, quais as suas características e configurações, métricas utilizadas no treinamento, descrever como o modelo foi treinado, apresentar quais foram as formas de avaliação}

\chapter{RESULTADOS E DISCUSSÕES}

\textit{Essa seção tem a missão apresentar o resultado da aplicação da metodologia, discussões, e ligação com o que foi proposto como objetivos do trabalho. Problema de Pesquisa e os Objetivos (Geral e Específicos) devem ser objetivamente respondidos pelo autor.}
\chapter{CONCLUSÃO}

\textit{Apresentação da opinião do autor a respeito da análise dos dados obtidos. Nesta etapa apresenta uma discussão final, ligando tudo o que foi feito desde o início e como os objetivos foram alcançados. O autor apresenta as limitações do estudo com relação ao problema, sugestões de modificações no método para futuros estudos.}


% ----------------------------------------------------------
% ELEMENTOS PÓS-TEXTUAIS
% ----------------------------------------------------------
\postextual

% Referências bibliográficas
\bibliography{referencias}

\end{document}